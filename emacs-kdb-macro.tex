\documentclass[russian,utf8]{beamer} 
\usepackage[T2A]{fontenc}
\usepackage[utf8]{inputenc}
\usepackage[english,russian]{babel}
\usepackage{graphicx}
\usepackage{beamerthemesplit}
\usetheme{Darmstadt}
\usecolortheme{dolphin}
\setbeamertemplate{footline}[text line]{}

\title{Клавиатурные макросы в Emacs}
\author{Попов~С.А.}
\date{25 января 2019}
\begin{document}
% \maketitle
\begin{frame}
  \frametitle{Клавиатурные макросы. Часть 2}
  \titlepage
\end{frame}

\begin{frame}
  \frametitle{Введение}
  \begin{block}{Почему Emacs?}
  \begin{itemize}
    \item<2-> редактор состоит из набора макросов (функции Lisp)
    \item<3-> функции можно привязать к любым клавиатурным сочетаниям
  \end{itemize}
  \end{block}
\end{frame}

% * Введение в клавиатурные макросы в Emacs
% ** Почему Emacs?
% ** Lisp-машина
% * Keys
%    1. C-x C-k C-e         - Edit the last defined keyboard macro
%    2. C-x C-k <RET>       - Edit the last defined keyboard macro
%    3. C-x C-k l           - Edit the last 300 keystrokes as a keyboard macro
%    4. C-u 0 C-x e         - Run macro unlimited
%    5. C-x C-k r           - Run macro in selected region
%    6. C-x C-k n           - Set name for last macro
%    7. M-x insert-kdb-macro - Insert kdb-macro as Lisp code
% * Пример создания множества файлов из одного файла.

\begin{frame}
  \frametitle{Создание и выполнение макроса}
  \begin{block}<2->{Простые команды для работы с клавиатурными макросами}
    \begin{itemize}
    \item C-x ( - начать запись макроса
    \item C-x ) - закончить запись макроса
    \item C-x e - запустить записанный макрос
    \end{itemize}
  \end{block}
  \begin{block}<3->{Сложные команды}
    \begin{itemize}
    \item C-x C-k <RET> - отредактировать последний макрос
    \item C-x C-k l - отобразить последние 300 команд
    \item C-u 0 C-x e - запустить макрос бесконечно
    \item C-x C-k r - запустить макрос на выделенном регионе
    \item C-x C-k n - дать имя макросу
    \item M-x insert-kdb-macro - получить Lisp-код макроса
    \end{itemize}
  \end{block}
\end{frame}

\begin{frame}
  \frametitle{Примеры}
\end{frame}

\begin{frame}
  Спасибо за внимание!
\end{frame}
\end{document}
%%% Local Variables: 
%%% mode: latex
%%% TeX-master: t
%%% End: 
